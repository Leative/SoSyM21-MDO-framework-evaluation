%\definecolor{blue}{RGB}{128, 177, 211}
\definecolor{redfill}{RGB}{251, 128, 114}
\definecolor{greenfill}{RGB}{140, 220, 130}
\definecolor{bluefill}{RGB}{128, 177, 211}
\definecolor{darkgreen}{RGB}{60, 190, 60}

% equiMark specifies a decoration which can be used to add equi-distant markers to a plot
\usetikzlibrary{decorations.markings}
\makeatletter
\tikzset{
  nomorepostactions/.code={\let\tikz@postactions=\pgfutil@empty},
  equiMark/.style 2 args={decoration={markings,
    mark= between positions 0 and 1 step (1/11)*\pgfdecoratedpathlength with{%
        \tikzset{#2,every mark}\tikz@options
        \pgftransformresetnontranslations \pgfuseplotmark{#1}%
      },  
    },
    postaction={decorate},
    /pgfplots/legend image post style={
        mark=#1,mark options={#2},every path/.append style={nomorepostactions}
    },
  },
}
\makeatother

% specify custom markers (rotated rectangles)
\newdimen\pgfplotmarkwidth
\newdimen\pgfplotmarkheight

\def\pgfsetplotmarkwidth#1{\pgfmathsetlength\pgfplotmarkwidth{#1}}
\def\pgfsetplotmarkheight#1{\pgfmathsetlength\pgfplotmarkheight{#1}}

\tikzoption{mark width}{\pgfsetplotmarkwidth{#1}}
\tikzoption{mark height}{\pgfsetplotmarkheight{#1}}

\pgfdeclareplotmark{rectRight}{%
	\pgftransformrotate{-35}
  \pgfpathrectangle
    {\pgfqpoint{-.5\pgfplotmarkwidth}{-.5\pgfplotmarkheight}}%
    {\pgfqpoint{\pgfplotmarkwidth}{\pgfplotmarkheight}}%
	\pgfpathrectangle
    {\pgfqpoint{.5\pgfplotmarkwidth}{-.5\pgfplotmarkheight}}%
    {\pgfqpoint{\pgfplotmarkheight}{\pgfplotmarkwidth}}%
  \pgfusepathqfillstroke
}

\pgfdeclareplotmark{rectLeft}{%
	\pgftransformrotate{35}
  \pgfpathrectangle
    {\pgfqpoint{-.5\pgfplotmarkwidth}{-.5\pgfplotmarkheight}}%
    {\pgfqpoint{\pgfplotmarkwidth}{\pgfplotmarkheight}}%
	\pgfpathrectangle
    {\pgfqpoint{.5\pgfplotmarkwidth}{-.5\pgfplotmarkheight}}%
    {\pgfqpoint{\pgfplotmarkheight}{\pgfplotmarkwidth}}%
  \pgfusepathqfillstroke
}

\newcommand\rectHeight{2.8pt}
\newcommand\rectWidth{1pt}
\newcommand\circleSize{1.2pt}
\newcommand\circleSizeApprox{1.6pt}

\tikzset{
	sc-mark/.style = {mark=x, blue, fill=bluefill, mark size=1pt, line width=0.3pt},
	sc-mark-equi/.style = {equiMark={x}{sc-mark}},
	uc-mark/.style = {mark=o, red, mark size=\circleSize, line width=0.3pt},
	uc-mark-equi/.style = {equiMark={o}{uc-mark}},
	sic-mark/.style = {mark=+, darkgreen, fill=greenfill, mark size=1.2pt, line width=0.3pt},
	sic-mark-equi/.style = {equiMark={+}{sic-mark}}
}
